%preámbulo
\documentclass[12pt,oneside,FLEQN]{report}
\usepackage{fancyhdr}
\usepackage[pages=some]{background}
\usepackage{amssymb,amsthm,amsmath,enumerate,graphicx,tabularx}
\usepackage[utf8]{inputenc}
\usepackage[hidelinks]{hyperref}
\usepackage[spanish]{babel}
\usepackage[rflt]{floatflt}
\usepackage{multicol}
\usepackage{tcolorbox, empheq}
\tcbuselibrary{skins,breakable,listings,theorems}
\usepackage{tikz,tkz-tab}
\usetikzlibrary{matrix,arrows, positioning,shadows,shadings,backgrounds,calc, shapes, tikzmark}
\usepackage{subfigure}
\usepackage{caption}
\usepackage[a4paper]{geometry}
\geometry{top=1.5cm, bottom=1.5cm, left=3cm, right=3cm}
\usepackage{ragged2e}
\newcommand{\marcar}[3]{\tikz[overlay,remember picture,baseline=-2pt] \node[circle,#1,draw,text=black, inner sep=1pt] (#2) { #3};}
\parindent=0cm
\usepackage{listings}
\hypersetup{
colorlinks=true,
linkcolor=black,
filecolor=magenta,
urlcolor=cyan,
citecolor=greenwhats
}
\backgroundsetup{
	scale=1,
	color=black,
	opacity=0.4,
	angle=0,
	contents={\includegraphics[scale=1.5]{Fondoportada.png}}
}
\usepackage[dvipsnames,table]{xcolor}
%\usepackage{lscape}
\usepackage{enumitem}
\definecolor{greenwhats}{RGB}{37, 211, 102}
\definecolor{gris}{rgb}{0.33, 0.41, 0.47}
\usepackage{fancyhdr}
\usepackage{natbib}
\usepackage{colortbl}
\usepackage{array,booktabs}
%\usepackage{pdflscape}
\usepackage{longtable}
\definecolor{codeturquoise}{RGB}{72,202,228}
\definecolor{codeyellow}{RGB}{255,170,51}
\definecolor{codepurple}{RGB}{255, 203, 242}
\definecolor{codegreen}{RGB}{149,213,178}
\definecolor{backcolour}{RGB}{73,80,87}
\definecolor{white}{RGB}{255,255,255}
\lstdefinestyle{estilochidori}{
backgroundcolor=\color{backcolour},
commentstyle=\color{codeyellow},
keywordstyle=\color{codeturquoise},
numberstyle=\tiny\color{codegreen},
stringstyle=\color{codepurple},
basicstyle=\ttfamily\footnotesize\color{white},
breakatwhitespace=false,
breaklines=true,
captionpos=b,
keepspaces=true,
numbers=left,
numbersep=5pt,
showspaces=false,
showstringspaces=false,
showtabs=false,
tabsize=2
}
\lstset{style=estilochidori}
\begin{document}
{
\fontfamily{qag}\selectfont
	\BgThispage
\begin{titlepage}
        \topmargin=1cm
        \centering

        {\bfseries\LARGE Universidad Autónoma de Querétaro \par}
        \vspace{1cm}
        {\scshape\Large  Facultad de Ingenier\'ia  \par}
        \vspace{3cm}
        %\centering
        \begin{figure}[!h]
        	\centering
                \includegraphics[height=5cm]{Logouaq.png}
        \end{figure}
        \vspace{2cm}
        {\itshape\large Parcial 2\par}
        \vspace{3cm}
        {\Huge Análisis numérico\par}
        \vspace{2cm}
        {\Large Autor: \par}
        {\large J.A. Salinas Sánchez \par}
        {\large Mayo 2022 \par}
\end{titlepage}
	\clearpage
	\newpage
\tableofcontents
\chapter{Solución}
	Este examen contiene 3 planteamientos que corresponden a 3 puntos de la valoración final. Resuelva los problemas de acuerdo a lo que estudió en clases. No olvide poner su código y capturas de pantalla mostrando su funcionamiento. Además, cuide que los resultados estén ordenados y que haya distinción entre cada uno de los problemas.
	\begin{center}
		Tabla de claificaciones de uso exclusivo para el profesor.\\
		\begin{tabular}{|c|c|c|c|c|}
			\hline
			Pregunta: &1&2&3&Total:\\
			\hline
			Puntos:&&&&\\
			\hline
			Resultado:&&&&\\
			\hline
		\end{tabular}
	\end{center}
	\section{Problema 1}
		1.(1 punto) Las estaciones de radar $A$ y $B$, separadas por una distancia $a=500m$, rastrean el avión $C$ grabando los ángulos $\alpha$ y $\beta$ en intervalos de 1 segundo. Si tres mediciones sucesivas son:\\
		\begin{center}
			\begin{tabular}{c|ccc}
				\hline
				$t(s)$&9&10&11\\
				\hline
				$\alpha$&54.80°&54.06°&53.34°\\
				$\beta$&65.59°&64.59°&63.62°\\
				\hline
			\end{tabular}
		\end{center}
		calcule la velocidad del avión y del ángulo de subida $\gamma$ a $t=10s$. Se puede mostrar que las coordenadas del avión son:
	\begin{center}
		\begin{tabular}{cc}
			$x=a\dfrac{\tan{\beta}}{\tan{\beta}-\tan{\alpha}}$\hfill&$y=a\dfrac{\tan{\alpha}\tan{\beta}}{\tan{\beta}-\tan{\alpha}}$\\
		\end{tabular}
	\end{center}
	\section{Problema 2}
	2.(1 punto) La siguiente tabla muestra la fuerza $F$ del arco en función del tirón $x$. Si el arco se tira $0.5m$, determine la velocidad de la flecha de $0.075kg$ cuando sale del arco. $Sugerencia$: utiliza el hecho de que $\dfrac{mv^{2}}{2}=\int_{0}^{x}Fdx$:
	\begin{center}
		{\small
		\begin{tabular}{c|ccccccccccc}
			\hline
			$x(m)$&0&0.5&0.10&0.15&0.20&0.25&0.30&0.35&0.40&0.45&0.50\\
			\hline
			$F(N)$&0&37&71&104&134&161&185&207&225&239&250\\
			\hline
		\end{tabular}
		}
	\end{center}
	\section{Problema 3}
	3.(1 punto) Ajuste la función $f(x)=axe^{x}$ a los siguientes datos y halle la desviación estándar:
	\begin{center}
		\begin{tabular}{c|cccc}
			\hline
			x&0.5&1.0&1.5&2.0&2.5\\
			y&0.541&0.398&0.232&0.106&0.052\\
		\end{tabular}
	\end{center}
}
\end{document}
