%preámbulo
\documentclass[12pt,oneside,FLEQN]{report}
\usepackage{amssymb,amsthm,amsmath,enumerate,graphicx,tabularx}
\usepackage[utf8]{inputenc}
\usepackage[hidelinks]{hyperref}
\usepackage[spanish]{babel}
\usepackage[rflt]{floatflt}
\usepackage{multicol}
\usepackage{tcolorbox, empheq}
\tcbuselibrary{skins,breakable,listings,theorems}
\usepackage{tikz,tkz-tab}
\usetikzlibrary{matrix,arrows, positioning,shadows,shadings,backgrounds,calc, shapes, tikzmark}
\usepackage{subfigure}
\usepackage{caption}
\usepackage[a4paper]{geometry}
\geometry{top=1.5cm, bottom=1.5cm, left=3cm, right=3cm}
\usepackage{ragged2e}
\newcommand{\marcar}[3]{\tikz[overlay,remember picture,baseline=-2pt] \node[circle,#1,draw,text=black, inner sep=1pt] (#2) { #3};}
\parindent=0cm
\usepackage{listings}
\hypersetup{
colorlinks=true,
linkcolor=black,
filecolor=magenta,
urlcolor=cyan,
citecolor=greenwhats
}
\usepackage[dvipsnames,table]{xcolor}
%\usepackage{lscape}
\usepackage{enumitem}
\definecolor{greenwhats}{RGB}{37, 211, 102}
\definecolor{gris}{rgb}{0.33, 0.41, 0.47}
\usepackage{fancyhdr}
\usepackage{natbib}
\usepackage{colortbl}
\usepackage{array,booktabs}
%\usepackage{pdflscape}
\usepackage{longtable}
\definecolor{codeturquoise}{RGB}{72,202,228}
\definecolor{codeyellow}{RGB}{255,170,51}
\definecolor{codepurple}{RGB}{255, 203, 242}
\definecolor{codegreen}{RGB}{149,213,178}
\definecolor{backcolour}{RGB}{73,80,87}
\definecolor{white}{RGB}{255,255,255}
\lstdefinestyle{estilochidori}{
backgroundcolor=\color{backcolour},
commentstyle=\color{codeyellow},
keywordstyle=\color{codeturquoise},
numberstyle=\tiny\color{codegreen},
stringstyle=\color{codepurple},
basicstyle=\ttfamily\footnotesize\color{white},
breakatwhitespace=false,
breaklines=true,
captionpos=b,
keepspaces=true,
numbers=left,
numbersep=5pt,
showspaces=false,
showstringspaces=false,
showtabs=false,
tabsize=2
}
\lstset{style=estilochidori}
\begin{document}
{
\fontfamily{qag}\selectfont
\begin{titlepage}
        \topmargin=0cm
        \centering

        {\bfseries\LARGE Universidad Autónoma de Querétaro \par}
        \vspace{1cm}
        {\scshape\Large  Facultad de Ingenier\'ia  \par}
        \vspace{2cm}
        \centering
        \begin{figure}[!h]
        \centering
                \includegraphics[height=5cm]{Logouaq.png}
        \end{figure}
        \vspace{3cm}
        {\itshape\large Tarea 7: Descomposición de Cholesky y Gauss-Seidel\par}
        \vspace{3cm}
        {\Huge Análisis numérico \par}
        \vspace{2cm}
        {\Large Autor: \par}
        {\large J.A. Salinas Sánchez \par}
        {\large Marzo2022 \par}
\end{titlepage}
\tableofcontents
\chapter{Introducción}
\chapter{Metodología}
	\section{Códigos}
		\subsection{Cholesky}
			\lstinputlisting{cholesky.py}
		\subsection{NR multivariable}
			\lstinputlisting{NRNL.py}
		\subsection{Gauss-Seidel}
			\lstinputlisting{gseidel.py}
	\section{Ejercicios}
		\subsection{Chapra}
			\subsubsection{11.3}
			Determine la matriz inversa del ejemplo 11.1 con base en la descomposición LU y los vectores unitarios:
			\begin{align}
				\begin{pmatrix}
					0.8&-0.4&0\\
					-0.4&0.8&-0.4\\
					0&-0.4&0.8
				\end{pmatrix}
				\begin{pmatrix}
					x_{1}\\
					x_{2}\\
					x_{3}
				\end{pmatrix}\\
				=\begin{pmatrix}
					41\\
					25\\
					105
				\end{pmatrix}
			\end{align}
			\subsubsection{11.5}
			Haga los mismos cálculos que en el ejemplo 11.2, pero para el sistema simétrico que sigue:
			\begin{align}
                                \begin{pmatrix}
					6&15&55\\
                                        15&55&255\\
					55&255&979
                                \end{pmatrix}
                                \begin{pmatrix}
                                        a_{1}\\
                                        a_{2}\\
                                        a_{3}
                                \end{pmatrix}\\
                                =\begin{pmatrix}
                                        152.6\\
                                        585.6\\
                                        2488.8
                                \end{pmatrix}
                        \end{align}
			\subsubsection{11.7}
			Calcule la descomposición de Cholesky de:
			\begin{align}
                                \begin{pmatrix}
                                        9&0&0\\
                                        0&25&0\\
                                        0&0&4
                                \end{pmatrix}
                        \end{align}
			\subsubsection{11.13}
			Use el método de Gauss-Seidel (a) sin relajación y (b) con relajación ($\lambda$ = 1.2), para resolver el sistema siguiente para una tolerancia de es = 5\%. Si es necesario, reacomode las ecuaciones para lograr convergencia.
		\begin{align}
                                \begin{pmatrix}
                                        2&-6&-1=-38\\
                                        -3&-1&7=-34\\
                                        -8&1&-2=-20
                                \end{pmatrix}
                        \end{align}
		\subsection{Kiusalaas}
			\subsubsection{23}
			Determine the coordinates of the two points where the circles $(x − 2)^2 + y^2 = 4$ and $x^2 + (y − 3)^2 = 4$ intersect. Start by estimating the locations of the points from a sketch of the circles, and then use the Newton-Raphson method to compute the coordinates.
			\subsubsection{24}
			Las ecuaciones:
			\begin{align}
				\sin{x}+3\cos{x}-2=0\\
				\cos{x}-\sin{y}+0.2=0
			\end{align}
\chapter{Conclusión}

}
\end{document}
